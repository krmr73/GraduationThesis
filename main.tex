%%
% このファイルは筑波大学情報学群情報科学類の卒業研究論文のサンプルです。
% このファイルを書き換えて、このサンプルと同様の書式の論文をLaTeXを使って
% 作成できます。
%
% OSやLaTeXの設定によっては漢字コードや改行コードを変更する必要があります。
%%
% \documentclass[a4paper,11pt]{jreport}

%%【PDF, PostScript, JPEG, PNG等の画像の貼り込み】
%% dvipdfmx を使う場合
\documentclass[uplatex,11pt,openany]{ujreport}

\usepackage[table]{xcolor}
\usepackage[dvipdfmx]{graphicx}

%% dvipdfmx を使ってPDFの「しおり」を付ける場合
%%\usepackage[dvipdfmx,bookmarks=true,bookmarksnumbered=true,bookmarkstype=toc]{hyperref} \usepackage{pxjahyper}

\usepackage{ulem}
\usepackage{times} % use Times font instead of default one
% \usepackage[super]{cite}
\usepackage{subcaption}

\usepackage{url}
\usepackage{caption}
\usepackage{booktabs}
\usepackage{amssymb}
\usepackage{lscape}
\usepackage{here}

\setcounter{tocdepth}{3}
\setcounter{page}{-1}
\renewcommand{\thesubtable}{\Alph{subtable}}

\setlength{\oddsidemargin}{0.1in}
\setlength{\evensidemargin}{0.1in}
\setlength{\topmargin}{0in}
\setlength{\textwidth}{6in}
%\setlength{\textheight}{10.1in}
\setlength{\parskip}{0em}
\setlength{\topsep}{0em}

%% タイトル生成用パッケージ(重要)
\usepackage{coins-jp}

%% タイトル
\title{ソーシャルネットワークにおける多様な相互作用を\\捉えるエージェントベースモデル}
%% 著者
\author{岩橋七海}
%% 指導教員 
\advisor{岡瑞起}

%% 年度と主専攻名
\fiscalyear{2024}
%\majorfield{ソフトウェアサイエンス主専攻}
% \majorfield{情報システム主専攻}
\majorfield{知能情報メディア主専攻}

\renewcommand{\bibname}{参考文献}

\begin{document}
\maketitle
\thispagestyle{empty}
\newpage

\thispagestyle{empty}
\vspace*{20pt plus 1fil}
\parindent=1zw
\noindent
%%
%% 論文の要旨
%%
\begin{center}
{\Large \bf 要  旨}
\vspace{2cm}
\end{center}
本研究では、人間社会の活動を理解するためにソーシャルネットワークにおける多様な相互作用をモデル化する。
既存の研究に基づき、ネットワークの成長を捉えるために「隣接可能空間」という概念に基づく新しいエージェントベースのモデルを提案する。
このモデルでは、既存の相互作用の強化、新しい相互作用の形成、隣接可能空間を広げるための紹介戦略を表すパラメータを用いて、ネットワークの成長とダイナミクスを表現する。

既存のモデルでは、相互作用した際に誰を相手のエージェントに紹介するかという戦略を事前に定義する必要があり、それが制約となっていた。
そこで、本研究では戦略をベクトルとして表すことで、より広い範囲を探索可能にする。
ベクトル表現によって大幅に増加した探索範囲を進化的アルゴリズムを用いることで効率的に探索する。
提案手法は既存モデルの全探索と同程度に優れた精度を示し、実社会のネットワークを精度良く捉えられることが示された。
また、提案手法によって実社会の相互作用は多様な戦略を取ることも明らかとなった。
オンラインサービスのソーシャルネットワークサービスやゲームのプラットフォームなどの実社会の振る舞いを再現し理解することで、効果的なサービスの開発や成長に貢献することが期待される。




%%%%%
\par
\vspace{0pt plus 1fil}
\newpage

\pagenumbering{roman} % I, II, III, IV
\tableofcontents
\listoffigures
%\listoftables

\pagebreak \setcounter{page}{1}
\pagenumbering{arabic} % 1,2,3

\chapter{序論}
    \section{背景}
    人間の社会は何気ない挨拶、取引や交渉、SNSでのコミュニケーションなどの様々な相互作用によって成り立っている。
    インターネットとモバイルデバイスの登場は、こうした多面的な相互作用に関するデータの記録と収集を容易にした。
    その中で、ソーシャルネットワークプラットフォームやオンラインソーシャルゲームを用いることで、人間の相互作用のモデリングや根的にあるパターンや原理の発見が可能になる。
    これらのコミュニケーションを包括的に理解することは、新規ユーザーを獲得し、既存のユーザーを維持するための効果的なサービスの開発に貢献すると考えられる。

    これまで人間の活動を再現し理解するために、数理モデルが用いられてきた。
    ネットワークは新たなノードやエッジの追加といった成長と、ネットワークの構造やエッジの重みといったダイナミクスの二つの側面から考えることができる。
    ネットワークの成長のモデルとしては、Barab\'{a}si--Albert (BA)モデルのような優先的選択性を捉え、スケールフリー性を表せるモデルがある\cite{barabasiEmergenceScalingRandom1999, barabasiOriginBurstsHeavy2005,barabasiScaleFreeNetworksDecade2009}。
    BAモデルはネットワークの成長を表すことはできるが、ネットワークの完全なダイナミクスを表現することはできない。
    現実世界のネットワーク構造をよりよく模倣するために、ネットワークのダイナミクスに焦点を当てた様々なモデルが提案されている。
    これらの方法には、1つのステップで複数のエッジを導入する方法\cite{albertTopologyEvolvingNetworks2000}、友人が繋がりを作る可能性が高い方法 (triadic closure\cite{bianconiTriadicClosureBasic2014})、ユーザーの職業や信頼性に基づくフィットネスなどのノードの特性を考慮する方法\cite{topirceanuWeightedBetweennessPreferential2018}などがある。
    しかし、これらのモデルは、既存のノード間の関係の強化や新しい接続を形成といった活動をとらえることができず、ネットワークの複雑さを完全に表現することはできない。

    これらのモデルの原理を活用し、実際のデータを再現する精度を向上させるために、様々なモデルが提案されている。
    その一つとして、新規性の出現は従来想定されていたようなランダムなものではなく、「隣接可能空間」から発生するという考え方を取り入れたものがある\cite{monechiWavesNoveltiesExpansion2017,sudaExplorationExploitationAdjacent2022, ubaldiEmergenceEvolutionSocial2021}。
    隣接可能空間とは現在の観測範囲から1ステップ外の空間のことであり、何らかのきっかけが一つあれば観測できる要素によって張られている。
    Kaufmanは当初、分子や生物の進化を解明するためにこの考えを提唱した\cite{kauffmanOriginsOrderSelforganization1993}。
    同様の考え方はタンパク質空間説にも見られ、遺伝子の進化は既存の遺伝子の微小な変化の積み重ねによって起こるものとし、その変化は遺伝子が表現型を形成できる制約の下で起こらなければならないと主張した\cite{maynardsmithNaturalSelectionConcept1970}。
    Kaufmanはこの理論を遺伝子の進化だけでなく、人間関係の進化や技術革新の発展など様々な分野に拡張し、一般化した。

    このような隣接可能空間の考え方は、Triaら\cite{triaDynamicsCorrelatedNovelties2014}によってモデル化された。
    このモデルはトリガー付き壺モデルと呼ばれ、環境に壺が一つ存在すると仮定し、壺からエージェントを表す玉を選択したり追加したりすることでシステムの動作を表す。
    Ubaldiらは、各エージェントにそれぞれの壺を割り当てることで、このモデルをエージェントベースのモデルに拡張した\cite{ubaldiEmergenceEvolutionSocial2021}。
    このモデルでは、隣接可能空間を通じて、エージェント同士のつながりを作ることで、社会的ネットワークが拡大すると仮定されている。
    エージェント個々の定義されたルールセットに従って、社会的相互作用のダイナミクスをカプセル化し、エージェントの行動を決定する。
    これらのルールは、ネットワークの成長とダイナミクスを捉え、既に相互作用したことがある人との相互作用、新しい人との相互作用、新しい人と関わるための隣接可能空間の探索を含む。
    比較的単純にもかかわらず、このモデルはTwitter上で行われる返信や論文の共著、携帯電話の発信などの振る舞いを正確に表現できることが報告されている\cite{ubaldiEmergenceEvolutionSocial2021}。

    しかし、隣接可能空間を探索するという考え方をモデルに組み込むことには課題もある。
    従来のモデルでは、人間がこの空間を探索する戦略をあらかじめ定義しておく必要がある。
    そのため、実際のデータに最も適合するパラメータの最適化探索は、これらの事前に定義した戦略に制約される\cite{sudaExplorationExploitationAdjacent2022,sudaAgentbasedModelUsing2022, ubaldiEmergenceEvolutionSocial2021}。
    この制限に対応するため、本研究では複数の戦略に対応し、これらを効率的に探索できるモデルを提案する。
    戦略をベクトル表現することで柔軟な探索を可能にし、進化的アルゴリズムのQuality Diversityアルゴリズムを用いることで効率的な探索を行う。
    本研究では提案モデルによって、実世界のソーシャルネットワークの相互作用を正確に捉えることを目的とする。

    提案モデルは、既存手法と比較して実データとの整合性において優れた性能を示した。
    また、似たネットワークの構造を持っていたとしても多様な戦略があり得ることが明らかとなった。
    モデルで振る舞いを正確に捉えることによって、実社会の相互作用を分析し、理解することができる。
    将来的には、より正確な予測モデルの設計からソーシャルメディア戦略の開発への情報提供まで、幅広い応用が期待される。



% BAモデルは成長するネットワークにおける優先的選択制を捉え、スケールフリーなネットワークの出現を説明できる。
% また、ネットワークのダイナミクスのモデルとしては、triadic closureを導入したモデルがあるが、既存のノード間の複数回の相互作用や既存のノード間の新しいエッジの追加などダイナミクスを考慮しきれていない。
% ポリアの壺モデルをベースとし、ネットワークの成長とダイナミクスの両方を考慮したエージェントベースモデルが提案されており、本研究はこのモデルをベースとしている。
% このモデルは様々な実データの振る舞いをより正確に表現できることが報告されている。


\chapter{関連研究}
    \section{エージェントベースのポリアの壺モデル}
    Ubaldiらのモデルは、社会的相互作用の複雑なダイナミクスを捉えるために、ポリアの壺モデルをベースとしている\cite{ubaldiEmergenceEvolutionSocial2021}。
    このモデルは社会的相互作用における個人を「エージェント」、個人がそれぞれ持つ情報空間を「壺」として表現する。
    この情報空間が、エージェントの隣接可能空間である。
    壺には今後相互作用を行う可能性のあるエージェント、すでに相互作用したエージェントを示す「玉」が入っている。
    このモデルは次の3ステップで相互作用を行う。
        \begin{description}
            \item[ステップ1] 発信側エージェントを選択する
            \item[ステップ2] 着信側エージェントを選択する
            \item[ステップ3] 発信側・着信側エージェント間で相互作用する
        \end{description}
    この3つのステップにより、相互作用のネットワークが生成される。
    これらのステップについて、以下で詳しく説明する。

        \subsection*{ステップ1: 発信側エージェントを選択する}
        このステップでは、環境全体から発信側エージェントを一人選択する。
        壺に入っている玉の数を重みとして、ランダムにエージェントを選択する。
        そのため、壺が大きいほど、そのエージェントが選択される確率が高くなる。
        これは、優先的選択制を考慮しており、コネクションの数が多いエージェントほど、新しいコネクションを構築する可能性が高くなることを表している。

        \subsection*{ステップ2: 着信側エージェントを選択する}
        ステップ1で決定した発信側エージェントが持っている壺の中から、ランダムに着信側エージェントを選択する。

        \subsection*{ステップ3: 発信側・着信側エージェント間で相互作用する}
        発信側と着信側の相互作用によって、両者の壺の状態が変化する。
        ここで、過去に相互作用しているエージェントとの相互作用のしやすさを表す$\rho$、新しいエージェントとの相互作用のしやすさを表す$\nu$、紹介するエージェントの選び方を表す戦略$s$の3つのパラメータが用いられる。
        Ubaldiのモデルの戦略には、Weighted Sample with Withdrawal (WSW)、Symmetric Sliding Window (SSW)、Asymmetric Sliding Window (ASW)などの6種類がある。

        3つのステップで相互作用の繰り返すことにより、ネットワークが成長する。
        その相互作用は図\ref{fig:model_ubaldi}の3つの状態のいずれかになり、それぞれ以下のように進行する。
        \begin{figure}[htbp]
            \centering
            \includegraphics[width=\textwidth]{img/model_ubaldi.png}
            \caption{相互作用の3つの状態}
            \label{fig:model_ubaldi}
        \end{figure}


            \subsubsection*{相互作用していない相手との相互作用(A)}
            発信側エージェントaと着信側エージェントbが初めて相互作用する場合、両者の壺にそれぞれ相手の玉を$\rho$個追加する。
            この強化ステップによって、既に相互作用したエージェントと再度相互作用する確率が変化し、$\rho$が大きいほど、再度相互作用する確率が高くなる。
            初めての相互作用の場合のみ、両者が持っている壺の中の玉を$\nu+1$個ずつ交換する。
            この紹介ステップによって、両者の隣接可能空間が広がり、過去に相互作用していない新しい人と相互作用する可能性ができる。
            $\nu$が大きいほど、相互作用したことのない新しいエージェントと相互作用する確率が高くなり、隣接可能空間が広がりやすい。
            壺の中から紹介するエージェントは戦略$s$によって決められる。

            \subsubsection*{既に相互作用した相手との相互作用(B)}
            発信側エージェントaと着信側エージェントbが再度相互作用する場合も両者の壺にそれぞれ相手の玉を$\rho$個追加する。
            初めて相互作用する相手でない場合は、互いの壺の中から交換する紹介ステップは行わないため、隣接可能空間は変化しない。

            \subsubsection*{新しいエージェントの追加(C)}
            どのエージェントとも繋がっていないエージェントが着信側として選ばれた場合、$\nu+1$個のエージェントをランダムに新しく壺に追加する。
            その後は相互作用していない相手との相互作用(A)の場合と同様に進行する。
            この場合は新しいエージェントが生成されるため、隣接可能空間をさらに広げることになる。







    \section{進化的アルゴリズム}
    進化的アルゴリズムは、生物の進化から着想を得た確率的探索手法である。
    問題に対して変化と選択に基づく世代交代を繰り返すことで、解の集団を進化させ最適解を得る。


        \subsection{遺伝的アルゴリズム (GA)}
        遺伝的アルゴリズムは、進化的アルゴリズムの中で最も一般的なアルゴリズムである。
        進化をモデルとした確率的探索手法の一つであり、問題の解を個体とする。
        個体ごとに遺伝子情報を与え、選択、交叉、突然変異といった操作を行うことで、個体の集団を進化させる。
        以下の流れで探索が行われる。
            \begin{enumerate}
                \item 初期集団を生成する
                \item 適応度を評価する
                \item 集団の中から親を選択する
                \item 親の遺伝子を交叉させる
                \item 子の遺伝子に突然変異を加える
                \item 2$\sim$5を繰り返す
            \end{enumerate}


        \subsection{Quality Diversityアルゴリズム}
        Quality Diversity (QD)アルゴリズムは、多様で高性能な解を探索するように設計されたアルゴリズムである\cite{pughQualityDiversityNew2016}。
        このアルゴリズムでは、個体の振る舞いを特徴づけるベクトルBehavioral Descriptor (BD)を利用して、領域内の解に対する個体の新規性を定量化する。
        各領域に対して一つの解を持ち、それぞれ可能な限り適合した解になるように反復的に決定される。
        QDには二つの代表的なアルゴリズムがある。

        一つ目はNovelty Search with Local Competition (NSLC)\cite{lehmanEvolvingDiversityVirtual2011}である。
        このアルゴリムは目的を持たせず、新規性のみを求める新規性探索アルゴリズム\cite{lehman_NoveltySearchProblemObjectives_2011}に局所的な競争の概念を導入している。
        個体をマッピングする領域は、BD空間におけるユークリッド距離に基づき、非構造化アーカイブを用いて決定する。

        二つ目の代表的なアルゴリズムはMuliti-dimensional Archive of Phenotypic Elites (MAP-Elites)\cite{mouretIlluminatingSearchSpaces2015}である。
        このアルゴリズムでは、BD空間はいくつかのグリッドに分けられる。
        各グリッドにはマッピングされた単一の最も適合した個体が保存される。
        そして、BD空間を埋めるように探索を進める。個体はグリッドから選択され、突然変異などのランダムな変化を受ける。
        そうしてできた新しい個体はその行動に対応するグリッドにマッピングされ、適合度が現在の占有者より高ければ新しい個体に置き換えられる。

        MAP-Elitesはセル数が特徴空間の次元数に応じて、指数関数的に増加するため、高次元のBD空間に対応するのは難しい。
        また、セル数が増えることによって、選択圧が低くなってしまうことも問題である。
        この問題を解決するために、MAP-Elitesを高次元の特徴空間にスケールアップするCVT-Map Elites\cite{vassiliadesUsingCentroidalVoronoi2017}が提案されている。
        CVT-MAP-Elitesは、BD空間をグリッド上ではなく、重心ボロノイ分割によってボロノイ状のセルに分割する。
        ボロノイ状のセルを採用することによって、セル数を固定しつつBD空間の次元を拡張することが可能になり、結果として高次元のBD空間の探索においてMAP-Elitesを上回る性能を示す。


\chapter{手法}
    既存のモデルでは、人間がこの空間を探索する戦略をあらかじめ定義しておく必要がある。
    そのため、実際のデータに最も適合するパラメータの最適化探索は、これらの事前に定義した戦略に制約される\cite{sudaExplorationExploitationAdjacent2022, sudaAgentbasedModelUsing2022,ubaldiEmergenceEvolutionSocial2021}。
    この制限に対応するため、本研究では複数の戦略に対応し、これらを効率的に探索できるモデルを提案する。
    戦略をベクトル表現することで柔軟な探索を可能にし、進化的アルゴリズムの QDアルゴリズムを用いることで効率的な探索を行う。
    提案するモデルによって、実世界のソーシャルネットワークの相互作用を正確に捉えることを目的とする。

    提案モデルでは、この問題を解決するために二つのことを行っている。
    (1)戦略をベクトル表現することで、柔軟な探索を可能にし、探索空間を拡張する。
    (2)進化的アルゴリズムの QDアルゴリズムを用いることで、拡張された探索空間に適した効率的な探索を行う。

    \section{戦略のベクトル表現}
        Ubaldiらによって提案された既存モデルでは、戦略sは事前に定義された方法に制限されていた。
        これらの戦略は実世界の相互作用の一部を捉えることができているが、実際の相互作用に用いられる多様な戦略を全て表現することはできない。
        例えば、頻繁に相互作用するが、最近相互作用していないエージェントを優先する戦略は、実世界のネットワークを正確に表現するかもしれない。
        このような細かい戦略を捉えるために、戦略を遺伝子のベクトルとして表現する。
        具体的には、戦略は最近度 (recency: r)、頻度 (frequency: f)という二つのパラメータで構成されるベクトルとして表され、パラメータは-1から1の範囲内で値を取る。
        例えば、最近度が-0.5であり、頻度が0.5の戦略ベクトルは、最近はあまり相互作用していないが、相互作用の頻度が高いエージェントを優先する。
        図\ref{fig:strategy}に提案手法の戦略ベクトルを使った相互作用の例を示す。
        図\ref{fig:strategy}-(A)は最近度が高いが頻度が低い相互作用を優先する戦略の場合を表しており、図\ref{fig:strategy}-(B)は最近度が低いが頻度が高い相互作用を優先する戦略の場合を表している。
        連続値で表現することで、探索空間のより包括的で柔軟な探索が可能になる。

        \begin{figure}[htbp]
            \centering
            \includegraphics[width=\textwidth]{img/model_proposed.png}
            \caption{提案手法の戦略をベクトルで表現した相互作用の例}
            \label{fig:strategy}
        \end{figure}

    \section{進化的アルゴリズムによる探索}
        Ubaldiらのモデルは、あらかじめ決められた範囲内でパラメータ$\rho,\nu,s$の組み合わせの全探索を行なっている。
        しかし、提案モデルでは戦略は連続値のベクトルとして表現されるため、潜在的な戦略の数が大幅に増加し、全探索するのは現実的ではない。
        この問題を解決するために、一つのパラメータセットを$\rho,\nu$と戦略を表す最近度$r$,頻度$f$で構成される4次元ベクトルの遺伝子として表し、進化的アルゴリズムを用いて実データに最も近いパラメータセットを効率的に探索する。

        具体的には、遺伝的アルゴリズムとQDアルゴリズムを用いる。
        遺伝的アルゴリズムでは、データとの距離$d$の逆数を適応度とし、選択にはルーレット選択を用いている。
        QDアルゴリズムとしては、高次元のBD空間に対してMAP-Elitesより高い性能を示すCVT-MAP-Elitesを用いる。
        提案モデルによって作成されたネットワークをBD空間にマッピングするために、graph2vecを用いて、ネットワークのグラフをベクトルに変換する。
        変換したベクトルをBDの軸として用いる。


    \section{評価指標}
        最適なパラメータはネットワークの特徴を表す10個の指標を用いて、定量的に評価される。
        モデルによって、実データを正確に表現することを目的とするため、モデルで生成したネットワークと実データのネットワークの指標の差を最小化するものを最適なパラメータとしている。
        モデルの目標データへの適応度を次式で計算される距離$d$を用いて評価する。
        \begin{equation}
            d = \sum^{p\in P}|P_{model}-P_{target}|
        \end{equation}
        ここで、$P={\gamma,C,Y,R, \langle h\rangle ,G,OC,OO,NC,NO}$は、ネットワークの成長とダイナミクスを表現する10個の指標の集合である。
        モデルによって作成した全てのデータの中の距離$d$が最小のものが最適なパラメータとなる。

        モデルの評価は、Ubaldiらが提案したモデルの生成するネットワークのトポロジーとそのダイナミクスに関する6つの指標\cite{ubaldiEmergenceEvolutionSocial2021}\footnote{Ubaldiらが提案している指標は8つだが、その中から詳細な記述と再現性が不足している指標を除いた6つの指標を用いている。}と、Monechiらが提案した4つの新規性に関する指標\cite{monechiWavesNoveltiesExpansion2017}を用いて行う。


        評価に用いた指標は以下の通りである。
        \subsection*{Heaps則の指数$\gamma$}
            Heaps則$\gamma$は成長するネットワークで新しい要素が発見されるペースを定量化する。
            \begin{equation}
                A(t) = \alpha t^{\gamma}
            \end{equation}
            ここで、$A(t)$はネットワークに張られるエッジ数、$t$は相互作用が発生した回数を示す。


        \subsection*{クラスター係数$C$}
            クラスター係数$C$は、ネットワークのクラスター性を表す。
            ネットワーク内のノード$i$のクラスター係数$C_i$は以下のように定義される。
            \begin{equation}
                C_i = \frac{2e_i}{k_i(k_i-1)}
            \end{equation}
            ここで、$e_i$はノード$i$のエッジ数、$k_i$はノード$i$の次数を示す。
            ネットワーク全体のクラスター係数$C$は、ノード$i$のクラスター係数$C_i$の平均として求められる。

        \subsection*{若さ係数$Y$}
            若さ係数$Y$は、相互作用の新しさを表す。
            長さ$T$のネットワークの成長時間ステップを一定幅$\Delta \tau$で分割することを考える。
            ある区間$I_i$の間に相互作用したノードの平均誕生ステップを$B(I_i)(1\leq i \leq \frac{T}{\Delta \tau})$とすれば、$I_i$と$B(I_i)$の関係をプロットできる。
            このプロットの傾きを$\lambda$とすると、若さ係数$Y$は以下のように定義される。
            \begin{equation}
                Y = \frac{\lambda}{\Delta \tau}
            \end{equation}


        \subsection*{最近度$R$}
            最近度$R$は各区間の相互作用の時間的な近さを表す。
            若さ係数と同様に、区間$I_i$を考え、$t_{max}(I_i)$を区間$I_i$において最も相互作用の多いノードの最初の出現ステップと考えると、最近度$R$は以下のように定義される。
            \begin{equation}
                R = \frac{\sum t_{max}(I_i)}{\sum i\Delta \tau}
            \end{equation}


        \subsection*{局所エントロピー$\langle h\rangle$}
            局所エントロピー$\langle h\rangle$は、相互作用するエージェントの多様性を示す。
            区間$I_i$に現れるエージェント$\alpha$の相互作用の数を$n_{\alpha}$、$f_{\alpha}=\frac{n_\alpha}{\tau}$としたとき、区間$I_i$の局所エントロピー$h_i$は以下のように定義される。
            \begin{equation}
                h_i = \frac{-\sum f_{\alpha} \log f_{\alpha}}{\log D(I_i)}
            \end{equation}
            ここで、$D(I_i)$は区間$I_i$における相互作用したエージェントの種類数を示す。
            ネットワーク全体の局所エントロピー$\langle h\rangle$は、区間$I_i$の局所エントロピー$h_i$の平均として求められる。

        \subsection*{疑似ジニ係数$G$}
            疑似ジニ係数、$G$はエージェントの注目度の格差を表す。
            エージェント$\alpha$の誕生順を$r_{\alpha}$、相互作用の頻度を$a_\alpha$とする。
            順位をエージェントの種類数$D$で正規化して、$x_\alpha=\frac{r_\alpha}{D}$を得る。
            また、$y_\alpha=\sum_{k=1}^{\alpha}a_k$とする。
            ここで、$x_\alpha$と$y_\alpha$の関係をプロットして得られた曲線と、直線$y=x$との間の面積が疑似ジニ係数$G$である。


        \subsection*{4つの相互作用の新規性と閉鎖性の指標}
            相互作用の新規性は、相互作用が以前に発生しているかどうかによって定義され、\verb|old|は複数回目の相互作用、\verb|new|は初めての相互作用を示す。
            閉鎖性の概念は相互作用するエージェント同士が、隣接可能空間の中で共通のエージェントがいるかどうかに関連している。
            \verb|closed|は共通のエージェントがいる場合、\verb|open|は共通のエージェントがいない場合を示す。
            これらの組み合わせにより、全ての相互作用は\verb|old-closed|($OC$)、\verb|old-open|($OO$)、\verb|new-closed|($NC$)、\verb|new-open|($NO$)の4種類に分類される。
            4種類の相互作用の出現頻度をネットワークの成長過程におけるトポロジーの指標とする。




\chapter{実験}
    本研究では3つのResearch Question (RQ)を設定しており、そのRQに対応する実験を行なった。
    以下にRQと実験の概要を示す。
        \subsubsection*{RQ1:提案手法はUbaldiらのモデルと比較して、モデルによって作成した合成データに対してどのような性能を示すか}
            RQ1に答えるためにランダムサーチ、遺伝的アルゴリズム、 QDアルゴリズムを用いた提案手法と、全探索を用いる既存手法を比較した。
            遺伝的アルゴリズムがランダムサーチよりも優れた利点を持つか確かめるために、ランダムサーチを用いている。
        \subsubsection*{RQ2:提案手法はUbaldiらのモデルと比較して、実世界のデータの振る舞いをどの程度正確に再現できるか}
            RQ2に答えるために、提案手法と既存手法を実世界のデータに適用し、その結果を比較した。
            モデルによって生成した相互作用と3つの実データの相互作用のネットワークの指標の一致度を評価することによって、提案手法の有効性を検証した。
        \subsubsection*{RQ3:実世界のデータにおける隣接可能空間の探索では、どのような戦略が用いられているか}
            RQ3に答えるために、実データに対して提案手法のQDアルゴリズムで見つけたパラメータから、隣接可能空間の探索に用いられている戦略を分析した。

    \section{データセット}
        RQ1に対する実験にはモデルによって作成した6つの合成データ、RQ2、RQ3に対する実験には実世界の振る舞いを示す3つの実データを用いた。
        \subsection{合成データ}
            Ubaldiのモデルでパラメータ$(\rho,\nu)\in {(5,5),(5,15),(20,7)}$、および$s\in{\texttt{SSW},\texttt{WSW}}$を設定した。
            ここで、SSWはSymmetric Sliding Window戦略を表し、壺の中に最も最近追加されたエージェントを優先する。
            一方、WSWはWeighted Sampling with Withdrawal戦略を表し、壺の中からランダムにエージェントを選択する。
            そのため、確率的に頻度が高いエージェントが選択されやすい。
            これらのパラメータ$\rho,\nu,s$を組み合わせた合計6 ($=3\times2$)つのパラメータセットを用いて、それぞれ20000ステップの相互作用を行い、生成されたネットワークを合成データとする。
            この合成データをUbaldiらのモデルと提案モデルのランダムサーチ、遺伝的アルゴリズムおよび QDアルゴリズムとの比較するための目標データとして用いた。

        \subsection{実データ}
            実データとしては、Twitter Mention Network (TMN)、American Physical Society (APS)共著ネットワークおよびMIXI Social Game Network (MIXI)の3つのデータセットを用いた。
            TMNとAPSのデータセットは、Ubaldiらの研究\cite{ubaldiEmergenceEvolutionSocial2021}で使用され、公開されている。
            TMNとAPSデータセットの両方から、20000のcaller-callee相互作用データを評価に使用する。
            MIXIデータセットは、日本の主要なゲーム会社である株式会社MIXIによって、提供していただいたオンラインソーシャルゲーム「モンスターストライク」のデータを用いている。
            MIXIデータセットから49,346,896のcaller-callee相互作用データを評価に使用する。

    \section{パラメータ}
        合成データと実データの両方において、最適なパラメータの探索を行った。
        各手法に対して、指定した範囲内のアルゴリズムのハイパーパラメータとパラメータ($\rho,\nu,s$)および($\rho,\nu,r,f$)を探索した。

        既存モデルでは、$1\leq \rho,\nu\leq20$および$s\in\{\texttt{SSW},\texttt{WSW}\}$の800 ($=20\times20\times2$)組のパラメータを全探索している。
        それぞれのパラメータセットで20,000ステップの相互作用を実行している。
        目標データとの距離$d$を最小に抑えたパラメータセット、すなわち目標データに最も近いネットワークを生成したパラメータセットを探索する。

        提案モデルでは、$1\leq \rho,\nu \leq30$および$-1.0\leq r,f\leq 1.0$の範囲で、ランダムサーチ、遺伝的アルゴリズム、QDアルゴリズムによりパラメータを探索する。
        ランダムサーチでは、合成データに対して100回、実データに対して500回の探索を行う。
        遺伝的アルゴリズム、QDアルゴリズムでは、合成データに対して100世代、実データに対して500世代に固定して探索を行う。
        遺伝的アルゴリズムでは、交叉率、突然変異率、個体数のハイパーパラメータをグリッドサーチにより最適化する。
        ハイパーパラメータは、交叉率$C\in\{0.80,0.85,0.90.0.95\}$、突然変異率$M\in\{0.01,0.02,0.03,0.04,0.05\}$、個体数$N\in\{10,20,30,40,50\}$の組み合わせの中から最適なものを選択した。

        QDアルゴリズムでは、Pythonライブラリの\texttt{pyribs}\cite{tjanakaPyribsBareBonesPython2023}を用いる。
        生成されたネットワークのグラフは、graph2vecアルゴリズムを用いて、ベクトルに変換する。
        CVT-MAP-Elitesのセル数とグラフを変換したベクトルの次元数をグリッドサーチにより最適化する。
        セル数は$250,500,750$、ベクトルの次元数は$64,128,256$の組み合わせの中から最適なものを選択した。
        変異率はCVT-MAP-Elitesでは初期値が設定されるが、CMA-ESアルゴリズム\cite{hansenReducingTimeComplexity2003}によって決められ、一定ではないためハイパーパラメータチューニングを行っていない。

        グリッドサーチによって、見つけた遺伝的アルゴリズムとQDアルゴリズムの最適なハイパーパラメータを表\ref{tab:hyperparameter}に示す。
        このパラメータを用いて、提案手法の探索を行った。

        \begin{table}[H]
            \caption{最適なハイパーパラメータ}
            \label{tab:hyperparameter}
            \centering

            \begin{subtable}[h]{\textwidth}
                \centering
                \caption{合成データ}
                \begin{tabular}{c|ccc|cc}
                    \hline
                                            & \multicolumn{3}{c|}{GA} & \multicolumn{2}{c}{QD} \\
                                            & 交叉率    & 突然変異率  & 個体数  & セル数      & BDの次元数      \\  \hline
                    ($\rho=5,\nu=5, s=$\texttt{SSW})  & 0.8    & 0.01   & 30   & 500      & 128         \\
                    ($\rho=5,\nu=5, s=$\texttt{WSW})  & 0.8    & 0.02   & 30   & 250      & 256         \\
                    ($\rho=5,\nu=15, s=$\texttt{SSW}) & 0.95   & 0.03   & 20   & 500      & 128         \\
                    ($\rho=5,\nu=15, s=$\texttt{WSW}) & 0.8    & 0.01   & 10   & 250      & 128         \\
                    ($\rho=20, \nu=7, s=$\texttt{SSW}) & 0.8    & 0.04   & 50   & 750      & 64          \\
                    ($\rho=20, \nu=7, s=$\texttt{WSW}) & 0.8    & 0.01   & 10   & 250      & 128         \\ \hline
                \end{tabular}
            \end{subtable}

            \vspace{1zh}

            \begin{subtable}[h]{\textwidth}
                \centering
                \caption{実データ}
                \begin{tabular}{c|ccc|cc}
                    \hline
                        & \multicolumn{3}{c|}{GA} & \multicolumn{2}{c}{QD} \\
                        & 交叉率   & 突然変異率   & 個体数   & セル数      & BDの次元数      \\ \hline
                    TMN  & 0.85   & 0.05    & 20    & 250      & 64         \\
                    APS  & 0.85   & 0.01    & 40    & 250      & 256         \\
                    MIXI & 0.85   & 0.04    & 30    & 250      & 128         \\ \hline
                \end{tabular}
            \end{subtable}

        \end{table}



    \section{実験結果と考察}
        \subsection{RQ1:合成データによる精度比較}
            6つの合成データのそれぞれに対して、既存モデルの全探索、提案モデルのランダムサーチ、遺伝的アルゴリズム、QDアルゴリズムを用いて、最適なパラメータを探索した。
            この実験では、正しいパラメータセットが既知の合成データに対するパラメータフィッティングによって精度を評価する。
            6つの合成データに対して実験を行ったが、全てのパラメータセットで同様の傾向が得られたため、QDアルゴリズムの距離が最小のパラメータセット$(\rho, \nu, s) = (5, 15, \texttt{SSW})$に対する結果のみを示す。

            図~\ref{fig:synthetic_boxplot}は、探索されたすべてのパラメータの距離$d$の最小値、最大値、四分位数を箱ひげ図で示している。
            \begin{figure}[H]
                \centering
                \includegraphics[width=\textwidth]{img/synthetic_box.png}
                \caption{パラメータ$(\rho,\nu,s)=(5,15,\texttt{SSW})$の合成データに対する各手法の距離の箱ひげ図の比較}
                \label{fig:synthetic_boxplot}
            \end{figure}


            図~\ref{fig:synthetic_boxplot}から、最小値はどのアルゴリズムでも大きな差はないことがわかる。
            また、距離の分散は既存手法と提案手法のランダムサーチが大きい。
            これらの手法は包括的な探索を行うため、距離の分散が大きくなる。
            次にQDアルゴリズム、遺伝的アルゴリズムの順に距離の分散が小さく、全体として距離が小さい。
            QDは高性能かつ多様な解が見つかっているため、遺伝的アルゴリズムと比較すると、距離の分散が大きいと考えられる。
            QDで見つかった多様な解については、後述するRQ3に対する結果で詳しく分析する。\\

            図~\ref{fig:synthetic_timeline}は、遺伝的アルゴリズムとQDアルゴリズムの100世代にわたる合成データ$(\rho, \nu, s) = (5, 15, \texttt{SSW})$を目標データとしたときの距離の変化を示す。
            波線は各世代において集団の中で最も最小の距離を示し、実線は集団全体の平均、薄い色で囲まれた領域は第一四分位数と第三四分位数を表している。

            \begin{figure}[H]
                \centering
                \includegraphics[width=\textwidth]{img/synthetic_timeline.png}
                \caption{パラメータ$(\rho,\nu,s)=(5,15,\texttt{SSW})$の合成データに対するGAとQDの探索の時間経過}
                \label{fig:synthetic_timeline}
            \end{figure}

            図\ref{fig:synthetic_timeline}より、どちらのアルゴリズムでも距離の値が世代ごとに減少し、学習が進んでいる。
            しかし、QDアルゴリズムの方が遺伝的アルゴリズムより収束が遅く、集団全体の平均の距離は大きい。
            これは、QDアルゴリズムがより多様な解を保つように探索するためだと考えられる。
            また、遺伝的アルゴリズムの結果において、平均の距離が前の世代より大幅に増える現象が時折見られる。
            これは突然変異によるものである。\\


            表~\ref{tab:best_synthetic_distance}は、それぞれの探索手法によって、見つけた最適なパラメータセットでの実行結果を示している。
            壺モデルの実行によって得られるネットワークにはランダム性があり、同じパラメータセットで実行したとしても、生成されるネットワークは全く同じにはならない。
            そのため、それぞれの手法で見つけた一つのパラメータセットを用いて壺モデルを10回実行し、その結果の平均と標準偏差を示した。

            \begin{table}[H]
                \centering
                \caption{最適なパラメータ値で生成したネットワークと合成データの距離}
                \label{tab:best_synthetic_distance}
                \begin{tabular}{cccc}
                \hline
                Existing    & QD          & GA          & Random      \\ \hline
                \textbf{0.068$\pm$0.026} & 0.209$\pm$0.059 & 0.198$\pm$0.045 & 0.161$\pm$0.025   \\   \hline
                \end{tabular}
            \end{table}

            既存手法による探索では、目標データとして用いる合成データのパラメータも探索されるため、最も高い適合度を示している。
            既存手法が示す距離の値は、壺モデル自体のランダム性によるものである。
            提案モデルでは、ランダムサーチが最良の結果であり、遺伝的アルゴリズムとQDもほぼ同等の結果を示した。
            QDアルゴリズムや遺伝的アルゴリズムの方がランダムサーチより効率的な探索が可能だが、この実験においてはわずか4次元のパラメータ空間を探索しているため、大きな差が出なかったと考えられる。\\


            生成したネットワークの10個の指標の結果を図~\ref{fig:radar_chart_synthetic}に示す。
            それぞれの手法で発見した最適なパラメータ値でモデルを10回の実行した結果の平均である。
            6つの合成データの中から、QDアルゴリズムで距離が最小のパラメータセット$(\rho, \nu, s) = (5, 15, \texttt{SSW})$と距離が最大のパラメータセット$(\rho, \nu, s) = (20, 7, \texttt{SSW})$の結果を示している。

            \begin{figure}[H]
                \centering
                \includegraphics[width=\textwidth]{img/diff_max_and_min.png}
                \caption{パラメータ$(\rho,\nu,s)=(5,15,\texttt{SSW})$の合成データに対する各測定指標の既存モデルと提案モデルの比較}
                \label{fig:radar_chart_synthetic}
            \end{figure}

            最良のケースでは既存手法の方が距離が小さかったが、評価指標を見ると、ごく僅かな差である。
            また、最悪のケースでも目標データから大きく外れていない。
            これらの結果から、提案手法は目標データの振る舞いを精度良く再現できている。


        \subsection{RQ2:実データによる精度比較}
            3つの実データに対して、既存手法と提案手法で最適なパラメータを探索し、実データをどの程度正確に再現できるかを評価した。

            図\ref{fig:boxplot}に、実データと各手法の距離$d$を箱ひげ図で示す。
            \begin{figure}[H]
                \centering
                \includegraphics[width=\textwidth]{img/boxplot.png}
                \caption{実データと各手法の距離の箱ひげ図の比較}
                \label{fig:boxplot}
            \end{figure}

            3つの実データにおいても、既存手法と提案手法のランダムサーチが距離の分散が大きく、距離の平均も大きい。
            次にQDアルゴリズム、遺伝的アルゴリズムの順に距離の分散が小さく、全体としても距離が小さい。
            合成データと同様の傾向を示している。\\


            図~\ref{fig:timeline}は、遺伝的アルゴリズムとQDアルゴリズムの500世代にわたる実データを目標としたときの距離の変化を示す。
            \begin{figure}[H]
                \centering
                \includegraphics[width=\textwidth]{img/timeline.png}
                \caption{実データに対するGAとQDの探索の時間経過}
                \label{fig:timeline}
            \end{figure}
            この学習曲線から、合成データと同様に両方のアルゴリズムで学習が進んでいることが確認できる。\\

            表\ref{tab:best_distance}は、それぞれの探索手法によって見つけた実データに対する最適なパラメータセットでの実行結果を示している。
            最適なパラメータ値で壺モデルを10回実行し、その平均と標準偏差を示した。
            \begin{table}[H]
                \centering
                \caption{最適なパラメータ値で生成したネットワークと実データの距離}
                \label{tab:best_distance}
                \begin{tabular}{ccccc}
                \hline
                     & Existing & QD & GA & Random \\ \hline
                TMN  & 0.837$\pm$0.025     & \textbf{0.763$\pm$0.056}       & 0.790$\pm$0.042       & 0.816$\pm$0.057   \\
                APS  & 0.603$\pm$0.037     & 0.520$\pm$0.028       & 0.534$\pm$0.033       & \textbf{0.518$\pm$0.024}   \\
                MIXI & \textbf{1.135$\pm$0.123}     & 1.178$\pm$0.081       & 1.209$\pm$0.068       & 1.213$\pm$0.069   \\ \hline
                \end{tabular}
            \end{table}
            提案手法はTMN, APSにおいて、既存手法よりも距離が小さく高い精度を示した。
            MIXIに対しては、既存手法の方が距離が小さかったが、ごく僅かな差である。
            これらの結果から、提案手法は実データの振る舞いを高い精度で再現できているといえる。\\

            図\ref{fig:radar_chart}は最も実データに適合するパラメータセットで生成したネットワークの10個の指標の結果を示す。
            \begin{figure}[H]
                \centering
                \includegraphics[width=\textwidth]{img/empirical_radar.png}
                \caption{実データに対する各測定指標の既存モデルと提案モデルの比較}
                \label{fig:radar_chart}
            \end{figure}
            提案手法は実データの軌道とよく一致していることがわかる。
            フィットしたパラメータと10個の指標から各ネットワークの特性を分析することができる。
            下記にモデルで見つけたパラメータとネットワークの10個の指標から解釈できる3つの実データの特徴について述べる。

            \subsubsection*{TMN}
            Twitter Mentionネットワークは、高い既存のつながりを強化するパラメータ($\rho = 16.79$)と高い新規性の強さを示すパラメータ($\nu = 25.30$)によって特徴づけられる。
            これらのパラメータの詳細はRQ3に対する結果で示す。
            モデルによって得られたこれらのパラメータから、相互作用したことのある既存つながりを強く強化し、新しい人とも相互作用しやすい動的に進化するネットワークであることが示唆される。
            また、紹介する相手を決める戦略パラメータは負の最近度パラメータ($r = -0.12$)と負の頻度パラメータ($f = -0.77$)を持ち、あまり最近交流しておらず、交流頻度が高くないノードに焦点を当てていることを示している。
            新しいノードが増える度合いが大きく急速な成長($\gamma = 0.99$)、ノードの頻出回数の格差を持つ($G = 0.50$)。
            つまり、一部のノードは多くの相互作用をしており(あるいは多くの他のノードに接続されており)、多くのノードはより少ない相互作用を持っている(あるいはより少ないノードに接続されている)、全体でより新しいノードがバランスよく注目され($Y = 0.50$)、
            各区間で新しいノードはやや注目されにくく($R = 0.39$)、相互作用するノードの多様性が高い($\langle h \rangle = 0.99$)ことを示している。
            また、共通の隣人を持つノードを巻き込む相互作用が少ない($NC = 0.0061$, $OC = 0.014$)。一方で、共通の隣人を持たないノードを巻き込む新しい相互作用が多い ($NO = 0.705$) パターンを示している。
            ネットワークには、共通の隣人を持たないノードを巻き込む古い相互作用もある($OO = 0.27$)。

            \subsubsection*{APS}
            APS共著ネットワークは、中程度の強化パラメータ($\rho = 9.55$)と高い新規性パラメータ($\nu = 28.59$)により特徴づけられ、既存のつながりの強く強化し、積極的に新しい人と相互作用する動的で進化するネットワークであることを示唆している。
            最近度の値が負($r = -0.47$)であり、頻度の値も負($f = -0.79$)であることから、あまり最近交流しておらず、交流頻度が高くないノードに焦点を当てていることが示されている。
            このネットワークは、新しいノードが増える度合いが大きい成長($\gamma = 0.43$)、全体として新規のノードがやや注目され($Y = 0.56$)、各区間でも同様に新規のノードはやや注目される($R = 0.54$)。
            そして相互作用の多様性が高いこと($\langle h \rangle = 0.99$)を示している。
            また、共通の隣接ノードを持つノード間の交流が少ない傾向($NC=0.0035$)($OC=0.0042$)と、共通の隣接ノードを持たないノード間での多くの新規交流($NO = 0.87$)が見られる。
            ネットワークには、共通の隣接ノードを持たないノード間の古い交流も適度に存在する($OO = 0.12$)。

            \subsubsection*{MIXI}
            MIXIのオンラインソーシャルゲームネットワークは、低い強化パラメータ($\rho = 2.90$)と低い新規性パラメータ($\nu = 2.05$)で特徴付けられ、既存の接続を均衡的に強化することに焦点を当てた、あまり動的でない成長するネットワークであると示唆される。最近度の値が負($r = -0.91$)であり、頻度の値が正($f = 0.34$)であることから、あまり最近でない相互作用と頻繁に交流するノードに焦点を当てていることが示されている。
            このネットワークは新しいノードが増える度合いが大きく急速な成長($\gamma = 0.86$)、ノードの頻出回数の大きな格差を持つ($G = 0.75$)。
            全体として新規のノードは注目されにくく($Y = 0.21$)、古い要素に強く焦点を当てる($R = 0.032$)、そして相互作用の多様性は高いこと($\langle h \rangle = 0.99$)を示している。
            また、共通の隣接ノードを持つノード間の交流が多く($NC = 0.14$および$OC = 0.35$)と、共通の隣接ノードを持たないノード間の新しい交流が適度にある($NO = 0.30$)ことも示されている。
            ネットワークには、共通の隣接ノードを持たないノード間の古い交流も多い($OO = 0.21$)。

            総じて、APSとTwitterネットワークはより動的に成長しており、既存のつながりを強く強化し、新たなノードとの潜在的なつながりを持ちやすい。
            一方、MIXIネットワークは、既存の接続を強めにくく、新たなノードとの潜在的なつながりも持ちにくい。
            そのため、あまり動的でない。
            ネットワーク指標の面では、APSとTwitterは類似したパターンを示すが、MIXIはノードの頻出回数が大きな格差を持ち、古い要素に焦点をより強く当て、共通の隣接ノードを持つノード間の交流が多い傾向がある。

            提案手法の戦略のベクトル化によって、既存手法よりも得られる情報が多くなり、ネットワークをより詳細に分析することができた。
            ベクトルに別の戦略を示すパラメータを追加して、より広範な戦略を探索するなど、この提案モデルをさらに拡張したモデルも考えられる。
            このような拡張は、実世界のデータで観察される振る舞いをより正確に表現し、ネットワークの特徴の深い理解につながる可能性がある。


        \subsection{RQ3:実データにおける隣接可能空間の探索に用いられている戦略}
            提案手法のQDアルゴリズムでは、探索して得られた$\rho, \nu, r, f$という4つのパラメータを分析することで、実データのネットワークの成長とダイナミクスを理解することができる。
            表\ref{tab:best-genes}は、提案手法によって見つけられた最適なパラメータの上位5つの詳細である。
            提案手法の中で、多くの高性能な解を得られるQDアルゴリズムの結果を示している。
            $\rho, \nu$の値の背景は緑、戦略を表す$r, f$の値の背景は赤であり、色が濃いほど大きい値を示す。

            \begin{landscape}
                \begin{table*}[tb]
                    \caption{提案手法のQDアルゴリズムによって見つけられた最適なパラメータの上位5つ}
                    \label{tab:best-genes}
                    \begin{subtable}[h]{0.3\textwidth}
                    \centering
                    \caption{TMN}
                    \label{table:best_genes_for_twitter}
                    \begin{tabular}{rrrrrr}
                    \toprule
                    $\#$ & d & $\rho$ & $\nu$ & $r$ & $f$ \\
                    \midrule
                    \color{black} 1 & \color{black} 0.656 & {\cellcolor[HTML]{ADE0B0}} \color[HTML]{000000} \color{black} 16.79 & {\cellcolor[HTML]{9CDAA0}} \color[HTML]{000000} \color{black} 25.30 & {\cellcolor[HTML]{FEC8C8}} \color[HTML]{000000} \color{black} -0.12 & {\cellcolor[HTML]{FFF1F1}} \color[HTML]{000000} \color{black} -0.77 \\
                    \color{black} 2 & \color{black} 0.660 & {\cellcolor[HTML]{AADFAE}} \color[HTML]{000000} \color{black} 17.20 & {\cellcolor[HTML]{9CDAA0}} \color[HTML]{000000} \color{black} 26.04 & {\cellcolor[HTML]{FDB0B0}} \color[HTML]{000000} \color{black} 0.28 & {\cellcolor[HTML]{FFF8F8}} \color[HTML]{000000} \color{black} -0.89 \\
                    \color{black} 3 & \color{black} 0.661 & {\cellcolor[HTML]{9FDBA3}} \color[HTML]{000000} \color{black} 19.34 & {\cellcolor[HTML]{9CDAA0}} \color[HTML]{000000} \color{black} 29.53 & {\cellcolor[HTML]{FEC8C8}} \color[HTML]{000000} \color{black} -0.10 & {\cellcolor[HTML]{FED6D6}} \color[HTML]{000000} \color{black} -0.33 \\
                    \color{black} 4 & \color{black} 0.664 & {\cellcolor[HTML]{B2E2B5}} \color[HTML]{000000} \color{black} 15.73 & {\cellcolor[HTML]{9CDAA0}} \color[HTML]{000000} \color{black} 23.23 & {\cellcolor[HTML]{FEE9E9}} \color[HTML]{000000} \color{black} -0.65 & {\cellcolor[HTML]{FDC0C0}} \color[HTML]{000000} \color{black} 0.02 \\
                    \color{black} 5 & \color{black} 0.666 & {\cellcolor[HTML]{ADE0B0}} \color[HTML]{000000} \color{black} 16.73 & {\cellcolor[HTML]{9CDAA0}} \color[HTML]{000000} \color{black} 26.70 & {\cellcolor[HTML]{FEE3E3}} \color[HTML]{000000} \color{black} -0.54 & {\cellcolor[HTML]{FED2D2}} \color[HTML]{000000} \color{black} -0.26 \\
                    \bottomrule
                    \end{tabular}
                    \end{subtable}\hfill\begin{subtable}[h]{0.3\textwidth}
                    \centering
                    \caption{APS}
                    \label{table:best_genes_for_aps}
                    \begin{tabular}{rrrrrr}
                    \toprule
                    $\#$ & d & $\rho$ & $\nu$ & $r$ & $f$ \\
                    \midrule
                    \color{black} 1 & \color{black} 0.442 & {\cellcolor[HTML]{D2EED4}} \color[HTML]{000000} \color{black} 9.55 & {\cellcolor[HTML]{9CDAA0}} \color[HTML]{000000} \color{black} 28.59 & {\cellcolor[HTML]{FEDEDE}} \color[HTML]{000000} \color{black} -0.47 & {\cellcolor[HTML]{FFF2F2}} \color[HTML]{000000} \color{black} -0.79 \\
                    \color{black} 2 & \color{black} 0.462 & {\cellcolor[HTML]{E1F4E2}} \color[HTML]{000000} \color{black} 6.80 & {\cellcolor[HTML]{9CDAA0}} \color[HTML]{000000} \color{black} 24.66 & {\cellcolor[HTML]{FEE5E5}} \color[HTML]{000000} \color{black} -0.59 & {\cellcolor[HTML]{FDBABA}} \color[HTML]{000000} \color{black} 0.13 \\
                    \color{black} 3 & \color{black} 0.463 & {\cellcolor[HTML]{DFF3E0}} \color[HTML]{000000} \color{black} 7.14 & {\cellcolor[HTML]{9CDAA0}} \color[HTML]{000000} \color{black} 23.62 & {\cellcolor[HTML]{FED9D9}} \color[HTML]{000000} \color{black} -0.39 & {\cellcolor[HTML]{FEC7C7}} \color[HTML]{000000} \color{black} -0.09 \\
                    \color{black} 4 & \color{black} 0.464 & {\cellcolor[HTML]{DBF2DD}} \color[HTML]{000000} \color{black} 7.84 & {\cellcolor[HTML]{9CDAA0}} \color[HTML]{000000} \color{black} 26.96 & {\cellcolor[HTML]{FED1D1}} \color[HTML]{000000} \color{black} -0.24 & {\cellcolor[HTML]{FDB5B5}} \color[HTML]{000000} \color{black} 0.20 \\
                    \color{black} 5 & \color{black} 0.466 & {\cellcolor[HTML]{D2EED3}} \color[HTML]{000000} \color{black} 9.75 & {\cellcolor[HTML]{9CDAA0}} \color[HTML]{000000} \color{black} 28.77 & {\cellcolor[HTML]{FEE6E6}} \color[HTML]{000000} \color{black} -0.60 & {\cellcolor[HTML]{FEE5E5}} \color[HTML]{000000} \color{black} -0.58 \\
                    \bottomrule
                    \end{tabular}
                    \end{subtable}\hfill\begin{subtable}[h]{0.3\textwidth}
                    \centering
                    \caption{MIXI}
                    \label{table:best_genes_for_mixi}
                    \begin{tabular}{rrrrrr}
                    \toprule
                    $\#$ & d & $\rho$ & $\nu$ & $r$ & $f$ \\
                    \midrule
                    \color{black} 1 & \color{black} 0.994 & {\cellcolor[HTML]{F5FBF6}} \color[HTML]{000000} \color{black} 2.90 & {\cellcolor[HTML]{FAFDFA}} \color[HTML]{000000} \color{black} 2.05 & {\cellcolor[HTML]{FFFAFA}} \color[HTML]{000000} \color{black} -0.91 & {\cellcolor[HTML]{FDADAD}} \color[HTML]{000000} \color{black} 0.34 \\
                    \color{black} 2 & \color{black} 1.024 & {\cellcolor[HTML]{F7FCF8}} \color[HTML]{000000} \color{black} 2.56 & {\cellcolor[HTML]{F6FCF7}} \color[HTML]{000000} \color{black} 2.67 & {\cellcolor[HTML]{FEDDDD}} \color[HTML]{000000} \color{black} -0.44 & {\cellcolor[HTML]{FEC2C2}} \color[HTML]{000000} \color{black} -0.01 \\
                    \color{black} 3 & \color{black} 1.030 & {\cellcolor[HTML]{F8FCF8}} \color[HTML]{000000} \color{black} 2.42 & {\cellcolor[HTML]{F5FBF6}} \color[HTML]{000000} \color{black} 2.87 & {\cellcolor[HTML]{FFEFEF}} \color[HTML]{000000} \color{black} -0.74 & {\cellcolor[HTML]{FD9B9B}} \color[HTML]{000000} \color{black} 0.62 \\
                    \color{black} 4 & \color{black} 1.033 & {\cellcolor[HTML]{F9FDF9}} \color[HTML]{000000} \color{black} 2.15 & {\cellcolor[HTML]{FAFDFA}} \color[HTML]{000000} \color{black} 2.01 & {\cellcolor[HTML]{FFFBFB}} \color[HTML]{000000} \color{black} -0.92 & {\cellcolor[HTML]{FEC7C7}} \color[HTML]{000000} \color{black} -0.08 \\
                    \color{black} 5 & \color{black} 1.034 & {\cellcolor[HTML]{F8FCF8}} \color[HTML]{000000} \color{black} 2.46 & {\cellcolor[HTML]{F8FCF8}} \color[HTML]{000000} \color{black} 2.40 & {\cellcolor[HTML]{FFFBFB}} \color[HTML]{000000} \color{black} -0.93 & {\cellcolor[HTML]{FFF4F4}} \color[HTML]{000000} \color{black} -0.82 \\
                    \bottomrule
                    \end{tabular}
                \end{subtable}\hfill\end{table*}
            \end{landscape}

            実データに正確に適合するパラメータセットでは、$\rho, \nu$のパラメータは全てのデータセットにおいて、それぞれ一貫した値が観察される。
            それに対して、提案手法での戦略を表す変数$r, f$はどのデータセットにおいても多様である。

            特にMIXIのデータで戦略の多様性が顕著であった。
            $r, f$の値は広範囲にわたり、最良の5つのパラメータの中に(-0.91, 0.34)と(-0.44, -0.01)という異なる値のペアがある。
            (-0.91, 0.34)の値を持つ戦略では、最近度の値-0.91はモデルがエージェントの重要性を決定する際に最近の相互作用を考慮する傾向が少ないことを示唆する。
            一方、頻度の値0.34は、頻繁に相互作用するエージェントを好むことを示している。
            したがって、この戦略は相互作用の最近度に関係なく、多くの相互作用が行われたエージェントを好むことになる。
            一方、(-0.44, -0.01)の値を持つ戦略は、やや異なる戦略を意味する。
            この場合、最近度と頻度の両方の値が負であり、最近相互作用しておらず、頻繁に相互作用していないエージェントを好むことを示唆する。
            この戦略は、既存の接続を強化するよりも、新しいエージェントやあまり相互作用したことのないエージェントを探索する。

            このように、似たネットワーク構造を持っていたとしても、その過程には多様な戦略が存在し得ることがわかった。
            これらの戦略の多様性は、提案手法のベクトル化された戦略によって明らかになった。
            \\

            QDアルゴリズムで発見したパラメータセットで作成したネットワークの多様性を検証するために、各セルの最良のパラメータセットを使用して生成されたネットワークから、graph2vecを用いて作成されたベクトルを可視化した。
            t-SNE法\cite{vandermaatenVisualizingDataUsing2008}を用いて可視化し、その結果を図\ref{fig:phenotype_map}-(A)\footnote{Map-Elitesアルゴリズムとは異なり、CVT-Map-Elitesは高次元のBDを扱うため、直接的な視覚化が不可能である。そのため、このアプローチを採用した。}に示す。
            図では、赤色は小さい距離でより高い精度を意味し、緑色は大きな距離でより低い精度を意味する。
            同様に、図\ref{fig:phenotype_map}-(B)は、遺伝的アルゴリズムによる最終世代で見つかったパラメータセットの可視化結果を示す。
            \begin{figure}[H]
                \centering
                \includegraphics[width=\textwidth]{img/phenotype_map.png}
                \caption{ベクトル化したQDとGAアルゴリズムによる探索で見つけたネットワークの可視化}
                \label{fig:phenotype_map}
            \end{figure}
            これらの結果から明らかなように、全てのデータセットで、QDアルゴリズムは遺伝的アルゴリズムと比較して、より多くの優れたパラメータセットを発見した。

            既存手法では多様な戦略を持つかどうかは不明であったが、提案手法により隣接可能空間内に存在する潜在的な相互作用の戦略の多様性が示された。
            既存手法は、戦略をあらかじめ定義する必要があり、単一の最適パラメータセットの探索に焦点を当てていた。
            一方、提案手法は戦略のベクトル表現とQDアルゴリズムによって、複数の優れたパラメータセットを探索し、相互作用の多様性を明らかにできる。


\chapter{結論}
    % 目的,方法
    本研究は数理モデルによって、実社会の人間の相互作用を正確に表現することを目指していた。
    Ubaldiらが提案したモデルでは、隣接可能空間を探索するという考え方をモデルに組み込むことで実データの振る舞いを精度良く再現できるが、探索範囲があらかじめ定義する戦略に制限されてしまうという課題があった。
    そこで、本研究ではより広い戦略を効率的に探索する手法を提案した。
    戦略を2つのパラメータのベクトルとして表現することで柔軟な探索を可能にした。
    また、戦略をベクトル化することによって起きる網羅的な探索が困難になる問題を、進化的アルゴリズムを用いて効率的に探索することで解決した。

    % 結果
    合成データ、実データを用いた実験により、提案手法はUbaldiらのモデルと同程度の精度で実データの振る舞いを再現できることが確認された。
    Twitter MentionネットワークやAPS共著ネットワークに代表されるソーシャルネットワークやMIXIのようなオンラインゲームのネットワークなど、実社会における様々な種類のネットワークを提案モデルで再現できた。
    % Barabási-Albertモデルのようなネットワーク成長モデルとは異なり、提案モデルは実社会のネットワークの様々な側面を同時に捉えることができる。
    エージェントの相互作用の最近度、頻度に重みを与えることで、相手に誰を紹介するかを決定する戦略を柔軟に表現でき、実社会に沿ったネットワークの成長を再現した。
    提案するフレームワークにおけるランダムサーチ、遺伝的アルゴリズム、QDアルゴリズムの比較により、QDアルゴリズムが実データの振る舞いを表現できる多様な戦略を発見できることが明らかになった。
    最終的に生成されるネットワークの構造が類似していたとしても、ネットワークの成長過程に用いられる戦略は多様であることが示された。

    % 意義と影響
    提案手法を用いて、精度良く実社会のネットワークを再現することにより、発見されたパラメータの値を分析し、ネットワークがどのような特性を持っているかを理解することができる。
    戦略がベクトルで表現されることにより、より細かい粒度での分析が可能になる。

    % 限界 今後の研究への提案
    一方で、提案手法ではUbaldiらの手法と同程度の精度が得られたが、大きく上回る精度ではなかった。
    パラメータが4つのみであったため、進化的アルゴリズムによる効率的な探索と網羅的な探索の差異が明確に現れにくかったと考えられる。
    また、モデルのランダム性が高いこともQDアルゴリズムの探索を困難にしている可能性がある。

    提案手法では、紹介するエージェントを決定する戦略を最近度、頻度の2つのパラメータで表現したが、実際にはこれら以外の特徴量も戦略に影響を与えている可能性がある。
    対象データの特徴に合わせて、エージェントの属性などをパラメータとして追加することで、より精度の高いモデルを構築できる。
    また、提案手法では発信側と着信側のエージェントはランダムに選択されている。
    優先的選択制が表現できているが、実際にはエージェントの属性によって決定する戦略など、他の戦略も考えられる。
    これらの戦略をベクトル表現することによって、より広範囲の探索が可能になり、高い精度で実データの振る舞いを再現できる可能性がある。
    また、パラメータを増やすことによって、発見されたパラメータからネットワークの特性をより詳細に分析できるようになる。
    さらに、パラメータが増えることでランダム性が小さくなり、QDアルゴリズムで最適なパラメータを探索しやすくなると考えられる。

    また、新たな側面からのアプローチとして、大規模言語モデルの生成エージェント\cite{parkGenerativeAgentsInteractive2023}を用いて、SNSの会話といったミクロな振る舞いをシミュレーションすることもできる。
    エージェントベースの壺モデルによる実社会に近いネットワーク構造で交流するエージェントを選択することで、よりリアルなシミュレーションができると見込まれる。










\chapter*{謝辞}
\addcontentsline{toc}{chapter}{\numberline{}謝辞}
    本研究の遂行にあたり、研究の着想から論文の完成に至るまで、熱心なご指導と多大なご助言を賜りました岡瑞起先生に深く感謝申し上げます。
    また、毎週のゼミや研究室での議論を通じて、様々なご助言をいただいた研究室の皆様にも心よりお礼申し上げます。
    特に、昨年ご卒業された須田幹大さんには、研究の方向性や実験の実装に関する多くのご助言をいただきました。
    須田さんの経験と知識から多くのことを学び、研究を進める上で大変お世話になりました。
    また、岡部純弥さんには実験に関するサポートとご助言をいただきました。
    さらに、株式会社MIXIの皆様には、実験に用いるデータセットを提供していただきました。
    このように、多くの方々のご協力に恵まれ、本研究を遂行することができました。
    心より感謝申し上げます。


\newpage

% \addcontentsline{toc}{chapter}{\numberline{}参考文献}
% \renewcommand{\bibname}{参考文献}

%% 参考文献に jbibtex を使う場合
% \bibliographystyle{junsrt}
% \bibliography{ref}


\bibliographystyle{junsrt}
\bibliography{ref}

% [compile] jbibtex sample; platex sample; platex sample;


\end{document}
