%%
% このファイルは、筑波大学情報学群情報科学類の
% 卒業研究中間報告用本体のサンプルです。
% このファイルを書き換えて、この例と同じような書式の論文本体を
% LaTeXを使って作成することができます。
%
% PC環境や、LaTeX環境の設定によっては漢字コードや改行コードを
% 変更する必要があります。
%%
\documentclass[uplatex,10pt]{ujarticle}

%%【PDF, JPEG, PNG等の画像の貼り込み】
%% 利用するパッケージを選んで下さい。
\usepackage[dvipdfmx]{graphicx} % for \includegraphics[width=3cm]{sample.pdf}
%\usepackage{epsfig} % for \psfig{file=sample.eps,width=3cm}
%\usepackage{epsf} % for \epsfile{file=sample.eps,scale=0.6}
%\usepackage{epsbox} % for \epsfile{file=sample.eps,scale=0.6}

\usepackage{times} % use Times Font instead of Computer Modern

\setcounter{tocdepth}{3}
\setcounter{page}{1}

\setlength{\oddsidemargin}{-.1in}
\setlength{\evensidemargin}{-.1in}
\setlength{\topmargin}{-4em}
\setlength{\textwidth}{6.5in}
\setlength{\textheight}{10in}
\setlength{\parskip}{0em}
\setlength{\topsep}{0em}
\setlength{\columnsep}{3zw}

%% タイトル生成用パッケージ(重要)
\usepackage{coins-jp}

%% タイトル
\title{ソーシャルネットワークにおける多様な相互作用を捉える\\エージェントベースモデル}
%% 著者
\gakuseki{202213544}
\author{岩橋七海}
%% 指導教員 
\advisor{岡瑞起}

%% 年度と主専攻名
\fiscalyear{2024}
%\majorfield{ソフトウェアサイエンス主専攻}
% \majorfield{情報システム主専攻}
\majorfield{知能情報メディア主専攻}

%% 提出日
\year{2023}
\month{1}
\day{ZZ}

\begin{document}
\abstract{
abstract
}
\maketitle

\section{序論}

\section{関連研究}

\section{手法}
\subsection{Ubaldiのモデル}
\subsection{提案モデル}
\subsubsection{戦略のベクトル表現}
\subsubsection{進化的アルゴリズムによる探索}

\section{実験}
\subsection{データセット}
\subsection{パラメータ}
\subsection{結果}

\section{結論}



%% 参考文献に jbibtex を使う場合
%\bibliographystyle{junsrt}
%\bibliography{samplebib}
%% [compile] jbibtex sample; platex sample; platex sample;

%% 参考文献を直接ファイルに含めて書く場合
\begin{thebibliography}{1}
\bibitem{Bibunsho}
奥村 晴彦, 黒木 裕介.
\newblock LaTeX2ε美文書作成入門 改訂第7版.
\newblock 技術評論社, 2017.

\bibitem{ScienceResearchWriting}
Hilary Glasman-Deal.
\newblock Science Research Writing: A Guide for Non-Native Speakers of English.
\newblock Imperial College Press, 2009.
\end{thebibliography}

\end{document}
